\documentclass{article}
\usepackage[UKenglish]{babel}
\usepackage[UKenglish]{isodate}
\usepackage{enumerate}
\usepackage[a4paper, inner=2cm, outer=2cm, top=2cm, bottom=2cm, bindingoffset=0cm]{geometry}

\def\cramersMatrix#1#2#3#4#5#6#7#8#9{
  \def\a{{#1}}
  \def\b{{#2}}
  \def\c{{#3}}
  \def\d{{#4}}
  \def\e{{#5}}
  \def\f{{#6}}
  \def\g{{#7}}
  \def\h{{#8}}
  \def\i{{#9}}
}

\def\cramersVector#1#2#3{
  \def\j{{#1}}
  \def\k{{#2}}
  \def\l{{#3}}
}

\newcommand{\cramers}[0]
{
\[
	\left(
		\begin{array}{ccc}
			\a & \b & \c \\
			\d & \e & \f \\
			\g & \h & \i
		\end{array}
	\right)
	\left(
		\begin{array}{c}
			x \\
			y \\
			z
		\end{array}
	\right)
	=
	\left(
		\begin{array}{c}
			\j \\
			\k \\
			\l
		\end{array}
	\right)
	\Rightarrow
	x=
	\frac
	{
		\left|
			\begin{array}{ccc}
				\j & \b & \c \\
				\k & \e & \f \\
				\l & \h & \i
			\end{array}
		\right|
	}
	{
		\left|
			\begin{array}{ccc}
				\a & \b & \c \\
				\d & \e & \f \\
				\g & \h & \i
			\end{array}
		\right|
	},
	y=
	\frac
	{
		\left|
			\begin{array}{ccc}
				\a & \j & \c \\
				\d & \k & \f \\
				\g & \l & \i
			\end{array}
		\right|
	}
	{
		\left|
			\begin{array}{ccc}
				\a & \b & \c \\
				\d & \e & \f \\
				\g & \h & \i
			\end{array}
		\right|
	},
	z=
	\frac
	{
		\left|
			\begin{array}{ccc}
				\a & \b & \j \\
				\d & \e & \k \\
				\g & \h & \l
			\end{array}
		\right|
	}
	{
		\left|
			\begin{array}{ccc}
				\a & \b & \c \\
				\d & \e & \f \\
				\g & \h & \i
			\end{array}
		\right|
	}
\]
}

\begin{document}
\title{Portfolio Excerise: Mathematics and Modelling for Games}
\author{Antony Bartlett}
\maketitle
\begin{enumerate}

% Question 1
\item

\begin{enumerate}[i.]

% Part i
\item The distance between the two points is
\[
\sqrt{ (a-c)^2 + (b-d)^2 }
\]
\\

% Part ii
\item A line can be described by the equation
\[
y=mx+c
\]

Put the gradient in
\[
y=\frac{d-b}{c-a}x+c
\]

At the point (a,b)
\[
b=\frac{d-b}{c-a}a + c
\]

Thus
\[
c=b - \frac{d-b}{c-a}a
\]

Hence the equation of the line is
\[
y=\frac{d-b}{c-a}x+b - \frac{d-b}{c-a}a
\]

Which simplifies to
\[
y=\frac{(d-b)(x-a)}{c-a} + b
\]
\\

% Part iii
\item The mid-point of the two points is
\[
(\frac{a+c}{2}, \frac{b+d}{2})
\]
\\

% Part iv
\item The gradient of the line joining the two points is
\[
\frac{d-b}{c-a}
\]
\\

% Part v
\item The gradient of the normal is
\[
\frac{a-c}{d-b}
\]
\\

% Part vi
\item The gradient of the tangent joining the two points is
\[
\frac{d-b}{c-a}
\]
\\

\end{enumerate}

% Question 2
\item Cramer's rule:
\cramersMatrix{a}{b}{c}{d}{e}{f}{g}{h}{i}
\cramersVector{j}{k}{l}
\cramers

\begin{enumerate}[i.]

% Part i
\item
\cramersMatrix{2}{3}{7}{3}{5}{1}{3}{4}{-2}
\cramersVector{12}{9}{5}
\cramers

\end{enumerate}

\end{enumerate}
\end{document}
