\documentclass{article}
\usepackage[UKenglish]{babel}
\usepackage[UKenglish]{isodate}
\usepackage{enumerate}
\begin{document}
\title{Portfolio Excerise: Mathematics and Modelling for Games}
\author{Antony Bartlett}
\maketitle
\begin{enumerate}
\item

% Question 1
\begin{enumerate}[i.]

% Part i
\item The distance between the two points is
\[
\sqrt{ (a-c)^2 + (b-d)^2 }
\]
\\

% Part ii
\item A line can be described by the equation
\[
y=mx+c
\]

Put the gradient in
\[
y=\frac{d-b}{c-a}x+c
\]

At the point (a,b)
\[
b=\frac{d-b}{c-a}a + c
\]

Thus
\[
c=b - \frac{d-b}{c-a}a
\]

Hence the equation of the line is
\[
y=\frac{d-b}{c-a}x+b - \frac{d-b}{c-a}a
\]

Which simplifies to
\[
y=\frac{(d-b)(x-a)}{c-a} + b
\]
\\

% Part iii
\item The mid-point of the two points is
\[
(\frac{a+c}{2}, \frac{b+d}{2})
\]
\\

% Part iv
\item The gradient of the line joining the two points is
\[
\frac{d-b}{c-a}
\]
\\

% Part v
\item The gradient of the normal is
\[
\frac{a-c}{d-b}
\]
\\

% Part vi
\item The gradient of the tangent joining the two points is
\[
\frac{d-b}{c-a}
\]
\\

\end{enumerate}

\end{enumerate}
\end{document}
