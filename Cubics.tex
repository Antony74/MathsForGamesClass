\documentclass{article}
\usepackage{enumerate}

\begin{document}
\title{Common Cubic Curves}
\date{}
\maketitle

These curves are constructed using just addition and multiplication.  However, because they require a *lot* of addition and multipication, they are summarised here in matrix notation.
\\

\begin{enumerate}

\item Cubic Bezier curves

\[
	P(t) =
	\left(
		\begin{array}{cccc}
			t^3 & t^2 & t & 1
		\end{array}
	\right)
	\left(
		\begin{array}{cccc}
			-1 &  3 & -3 & 1  \\
			 3 & -6 &  3 &  0 \\
			-3 &  3 &  0 &  0 \\
			 1 &  0 &  0 &  0 
		\end{array}
	\right)
	\left(
		\begin{array}{c}
			p_1      \\
			p_{c1} \\
			p_{c2} \\
			p_2
		\end{array}
	\right)
\]
\\

This defines a curve from $p_1$ (at t = 0) to $p_2$ (at t = 1), using control points $p_{c1}$ and $p_{c2}$.
The other curves are splines - which means you can join a series of points with a series of curves, and the whole thing
will be nice and smooth.  The curve $P_i(t)$ runs from $p_i$ (at t=0) to $p_{i+1}$ (at t = 1).  $p_{i-1}$ is where the previous curve starts and $p_{i+2}$ is where the next curve ends.  Beziers can easily be splined by offsetting $p_{c1}$ from $p_1$ by the same amount and in the opposite direction from which the previous curve offset $p_{c2}$.

\item Uniform cubic B-splines

\[
	P_i(t) = 
	\left(
		\begin{array}{cccc}
			t^3 & t^2 & t & 1
		\end{array}
	\right)
	\frac{1}{6}
	\left(
		\begin{array}{cccc}
			-1 &  3 & -3 & 1  \\
			 3 & -6 &  3 &  0 \\
			-3 &  0 &  3 &  0 \\
			 1 &  4 &  1 &  0 
		\end{array}
	\right)
	\left(
		\begin{array}{c}
			p_{i-1}  \\
			p_{i}     \\
			p_{i+1} \\
			p_{i+2} \\
		\end{array}
	\right)
\]

\item Catmull Rom splines

\[
	P_i(t) = 
	\left(
		\begin{array}{cccc}
			t^3 & t^2 & t & 1
		\end{array}
	\right)
	\left(
		\begin{array}{cccc}
			0 & 0 & 1 & 0 \\
			0 & \tau & 0 & -\tau \\
			-\tau & 3 - 2\tau & \tau - 3 & 2\tau
		\end{array}
	\right)
	\left(
		\begin{array}{c}
			p_{i-1}  \\
			p_{i}     \\
			p_{i+1} \\
			p_{i+2} \\
		\end{array}
	\right)
\]

0.5 is said to be a good value for the curve-tightness $\tau$ to start with when playing with Catmull Rom splines.

\end{enumerate}
\end{document}
